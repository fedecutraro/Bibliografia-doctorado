% !TEX root = ./main.tex
% !TEX outputDirectory = ./Build/
\chapter{Electrificación}

\section{Modelado}

\paragraph{Fierro y Reisner 2011} \label{Fierro y Reisner 2011}
\textbf{High-Resolution Simulation of the Electrification and Lightning of Hurricane Rita during the Period of Rapid Intensification}

La tasa de rayos en el ojo de los huracanes es muy baja exceptuando los momentos previos a fases de rápida intensificación. En este trabajo se proponen investigar qué tan bien ven los modelos los eventos convectivos que derivan en la rápida intensificación del huracán Rita. Hacen una simulación con 2 km de resolución para la cual inicializan a partir de una de 4 km en la que asimilan además de otras variables, la actividad eléctrica via nudging a partir de saturar entre 3 y 11 km de altura los puntos de retícula donde se observaron rayos. Logran ver en las simulaciones un incremento de la actividad eléctrica y los eventos convectivos previo a la intensificación del huracán. Hacen correlaciones entre la actividad eléctrica pronosticada por la simulación y distintas variables de pronóstico y teorizan con asimilar la actividad eléctrica usando nudging al saturar capas de atmósfera o añadiendo calor latente.
