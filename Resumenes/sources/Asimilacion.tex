% !TEX root = ./main.tex
% !TEX outputDirectory = ./Build/
\chapter{Asimilación}


%%%%%%%%%%%%%%%%%%%%%%%%%%%%%%%%%
%% Metodologías de asimilación %%
%%%%%%%%%%%%%%%%%%%%%%%%%%%%%%%%%
\section{Metodologías}


%%%%%%%%%%%%%%%%%%%%%%%%%
%% Latent Heat Nudging %%
%%%%%%%%%%%%%%%%%%%%%%%%%
\subsection{Latent Heat Nudging}

\paragraph{Jones y MacPherson 1996} \label{Jones y MacPherson 1996}
\href{https://rmets.onlinelibrary.wiley.com/doi/pdf/10.1017/S1350482797000522}{\textbf{A latent heat nudging scheme for the assimilation of precipitation data into an operational mesoscale model}}

Aplican el Latent Heat Nudging para asimilar tasas de precipitación estimadas por radar en el Reino Unido. La forma en que lo aplican es al comparar la tasa de precipitación con la del modelo y sacan un factor con el que escalan el valor del calor latente en el modelo. Este factor no puede ser mayor a 3 o menor a 1/3, es decir no se puede aplicar si la PP observada es 3 veces mayor que la pronosticada o al revés. En los casos en que la observación es mucho mayor que el pronóstico se busca un perfil cercano en el pronóstico que tenga valores más parecidos para no introducir demasiado calor, en caso de que no lo haya no se aplica la metodología. Para eliminar posibles inestabilidades suavizan el campo del incremento. Encuentran que en el promedio de los 14 casos en que aplicaron la metodología los scores le mejoran incluso hasta las 9 horas de pronóstico respecto al control. Hacen una evaluación subjetiva en la que llegan a que la mejora sobre el pronóstico es leve pero que depende mucho del caso en cuestión.


\section{Asimilación de rayos (LDA)}

\paragraph{Jones y MacPherson 1997} \label{Jones y MacPherson 1997}
\href{https://library.metoffice.gov.uk/Portal/DownloadImageFile.ashx?fieldValueId=1189}{\textbf{Sensitivity of the limited area model to the assimilation of precipitation estimates derived from lightning data}}

Este trabajo es una continuación de \nameref{Jones y MacPherson 1996} en la que intentan asimilar tasas de precipitación estimadas a partir de rayos. Usan el sistema Arrival Time Difference (ATD, \textcolor{red}{Lee 1988}) para obtener la ubicación de los rayos. Para el tiempo del estudio la red no era muy buena y tenía una eficiencia de detección de alrededor de 10\%. La relación que usan para pasar de rayos a PP es $R = 4.3F$ donde $R$ es la tasa de precipitación y $F$ la cantidad de flashes cada 15 minutos en un área de 10 $km^{2}$, esta relación fue calculada para Florida y podría no andar del todo bien en el Reino Unido. Realizan varios experiemntos para ver la sensibilidad del sistema, como usar la mitad del valor de la estimación de precipitación o asignarle un valor arbitrario de $2 \frac{mm}{hr}$ para considerar solo errores en la ubicación de la precipitación y en ese caso no corrigen "hacia abajo". No encuentran un resultado muy significativo, hay casos en que anda bien y otros en que no. Detallan varias problemáticas de la metodología como que solo provee información donde hay rayos, asociado a esto el problema de que no elimina la convección espúera y ante errores de posición se tendría a extender el área de precipitación más que a "moverla" generando un bias en el pronóstico.

\paragraph{Alexander et al. 1999} \label{Alexander et al. 1999}
\href{https://journals.ametsoc.org/view/journals/mwre/127/7/1520-0493_1999_127_1433_teoarr_2.0.co_2.xml}{\textbf{The Effect of Assimilating Rain Rates Derived from Satellites and Lightning on Forecasts of the 1993 Superstorm}}

Asimilan vapor de agua integrado y tasa de precipitación derivada de imágenes satelitales de IR y MW y de la tasa de rayos en forma de calor latente para el caso de una "supertormenta" del año 1993 en EEUU que los modelos no pudieron predecir su rápida intensificación. Explican detalladamente como fusionaron todas las fuentes de información para generar un continuo de datos de tasa de PP. El modelo que usan es el MM5 versión 1 con una resolución de 40 km y 23 niveles verticales. Como observaciones de rayos usan las de la National Lightning Detection Network (NDLN, \textcolor{red}{Cummins et al. 1998}) y como esta red funciona solo en el CONUS la extienden con la red de la United Kingdom Meteorological Office (\nameref{Lee 1986b}) pero que tiene menos tasa de detección. Para asimilar el IWV usan nudging y para la tasa de precipitación usan el esquema de \textcolor{red}{Karyampudi et al. 1998}. Se hacen pruebas en las que se usan solo datos de MW, MW + IW y MW + IW + rayos. Los resultados indican que a medida que se añadieron más observaciones el error en el pronóstico es menor y la convección observada está mejor representada. Les mejora la ubicación de la precipitación pero los máximos están sobrestimados. Llegan a la conclusión de que la simulación que asimila rayos es la que mejor da porque genera más convección que las otras y el sistema se intensifica más.

\paragraph{Chang et al. 2001} \label{Chang et al. 2001}
\href{https://journals.ametsoc.org/view/journals/mwre/129/8/1520-0493_2001_129_1809_teosma_2.0.co_2.xml}{\textbf{The Effect of Spaceborne Microwave and Ground-Based Continuous Lightning Measurements on Forecasts of the 1998 Groundhog Day Storm}}

Hacen un estudio similar al de \nameref{Alexander et al. 1999} asimilando IWV y rayos a través de convertirlos a tasa de precipitación para un caso de una tormenta intensa. El caso es interesante porque la mayor parte de la convección se da sobre el Golfo de México y por lo tanto hay pocas observaciones, usar rayos podría beneficiar bastante. Para pasar de tasa de precipitación a rayos hacen un histogram matching entre las estimaciones de precipitación \textbf{convectiva} del TMI del TRMM y la tasa de rayos medidas por la red STARNET-1. Para asimilar los rayos usan un esquema similar al del \nameref{Jones y MacPherson 1996} poniendo como límites para el factor de corrección de $-\frac{2}{3}$ y 3. Para asimilar el IWV usan nudging con la metodología de \textcolor{red}{Kuo et al. 1993}. En los resultados se ve para la hora del análisis que el experimento que asimila es el que mejor anda, representando bien la dinámica del entorno y la precipitación, cosa que no pasa para el control o el que no asimila alguna de las 2. Usar el campo de SST estimado por el TRMM también representa una mejora a usar el del NCEP. Los buenos resultados se observan también para el pronóstico a 18 hs, tanto la precipitación como la presión en superficie están mejor representadas en el experimento que asimila. Hacen también unos experimentos en lso que escalan los valores de precipitación que van a asimilar y llegan a que poner valores bajos claramente tiende a la corrida control pero excederse mucho (duplican la precipitación) no afecta mucho.

\paragraph{Papadopoulos et al. 2005} \label{Papadopoulos et al. 2005}
\href{https://journals.ametsoc.org/view/journals/mwre/133/7/mwr2957.1.xml}{\textbf{Improving Convective Precipitation Forecasting through Assimilation of Regional Lightning Measurements in a Mesoscale Model}}

Usan datos de tasa de rayos para asimilar via nudging perfiles de humedad que ayuden a la parametrización de la convección a generar mejores perfiles de calor latente. En los casos en que ya hay convección profunda en el modelo donde hay rayos realizan un ajuste a un perfil de humedad empírico, en los casos en que la convección no es profunda o está ausente aplican un proceso iterativo en el que van ajustando el perfil hasta que el modelo resuelve convección profunda o se itero 3 veces. Para no generar inestabilidades en el modelo imponen condiciones sobre la constante del término de nudging, el incremento aplicado y que no se puede llegar a un estado sobresaturado. Los perfiles empíricos los obtienen de sondeos de entornos convectivos y mediante prueba y error definen uno que es el que mejor resultados les dio. Encuentran que les mejora la precipitación para las primeras horas especialmente. Las mayores mejoras se dan para los valores altos de precipitación, lo que tiene sentido porque ahí es donde deberían estar los rayos.

\paragraph{Pessi y Cherubini 2006} \label{Pessi y Cherubini 2006}
\textbf{Comparison of Two Methods for Assimilation of Lightning Data into NWP Models}

Asimilan datos de la red Pacific Lightning Detection Network (PacNet) en el modelo MM5 con una resolución de 27 km usando una relación entre la tasa de rayos y la precipitación convectiva. Para obtener la relación comparan la tasa de rayos medidos por la red PacNet con la precipitación convectiva medida por los sensores montados en los satélites Aqua y TRMM para varios casos. Los datos los asimilan de 2 maneras distintas, una en la que usan perfiles de humedad relacionados a cada valor de precipitación convectiva que los asimilan usando nudging y otro igual al empleado en \nameref{Alexander et al. 1999} y \nameref{Chang et al. 2001}.  En ambos casos se corrieron pronósticos a 24 horas donde se aplicó el nudging en las primeras 8. Evalúan las 2 metodologías en 2 casos observando mejoras en ambos, tanto en la ubicación de los sistemas como en su intensidad.

\paragraph{Mansell et al. 2007} \label{Mansell et al. 2007}
\href{https://journals.ametsoc.org/view/journals/mwre/135/5/mwr3387.1.xml}{\textbf{A Lightning Data Assimilation Technique for Mesoscale Forecast Models}}

Este trabajo va en una línea similar a la de \nameref{Papadopoulos et al. 2005} porque asimilan la actividad eléctrica empleando la parametrización de la convección. El objetivo es mejorar la distribución de las piletas de aire frío en los análisis. En este caso modifican la parametrización para que se active la función de trigger en las zonas de rayos. El funcionamiento es el siguiente, si la parametrización no activa convección pero la tasa de rayos está por encima de un cierto valor se fuerza la activación de la misma, si la tasa de rayos se encuentra por debajo del umbral la parametrización se mantiene apagada o se la limita. En los casos en que el modelo no da convección se incrementa la humedad en $1\frac{gr}{kg}$ como máximo hasta cumplir un criterio de espesor de nube e intensidad de la ascendente. Toman un solo caso de estudio en el que asimilan durante 1 día en el que hubo convección y evaluan el impacto en el pronóstico del día siguiente en el que también se desarrolló convección. Les mejora bastante la precipitación durante el spinup, incluso en los casos en que no activan la supresión de convección espúrea, la que aparece en la corrida control no se ve en las que asimilan. Para el final del ciclo de asimilación las piletas de aire frío están ubicadas correctamente en las corridas que asimilan. En los pronósticos también se ven mejoras en cuanto a la ubicación de los máximos de precipitación pero estos son bastante menores que los observados y en la evolución de las tormentas en las primeras horas.

\paragraph{Hakim et al. 2008} \label{Hakim et al. 2008}
\href{https://www.vaisala.com/sites/default/files/documents/Lightning%20Data%20Assimilation%20using%20an%20Ensemble%20Kalman%20Filter.PDF}{\textbf{Lightning Data Assimilation using an Ensemble Kalman Filter}}

Usan datos de la red NLDN para asimilarlos via filtro de Kalman en el WRF. Generan análisis cada 6 horas durante un período de 2 semanas y cada 12 horas sacan pronósticos a 24 y 48 horas. El ensamble es de 90 miembros y la resolución es de 45 km. Asimilan los datos de rayos conviertiéndolos en tasa de precipitación convectiva como se hizo en \nameref{Pessi y Cherubini 2006} de 2 maneras diferentes, asimilando todas las tasas de rayos y otra en la que primero les hacen un grillado. Estos 2 experimentos más el control los hacen para 3 casos de profundización de un ciclón extratropical, uno en el que hubo una gran tasa de rayos y otros dos donde no. En el que hubo gran tasa de rayos el ciclón presenta menor presión a nivel del mar respecto del control y más acordes con los análisis del GFS y el NCEP.En los casos de poca actividad eléctrica también se ven mejoras indicando que unas pocas observaciones pueden generar un impacto positivo en los análisis.

\paragraph{Weygandt et al. 2008} \label{Weygandt et al. 2008}
\href{https://www.vaisala.com/sites/default/files/documents/Assimilation%20of%20Lightning%20Data%20Using%20a%20Diabatic%20Digital%20Filter%20Within%20the%20Rapid%20Update%20Cycle.PDF}{\textbf{Assimilation of lightning data using a diabatic digital filter within the Rapid Update Cycle}}

Usan los datos de rayos para suplementar la información provista por los radares para asimilarlos en el RUC. Estas 2 fuentes de información las transforman a tasa de calor latente al asimilarlas. La tasa de rayos la convierten a reflectividad a partir de una relación simple y acumulan rayos durante 30 minutos antes y 10 después de la hora del análisis. Este valor lo usan donde no hay cobertura de radares o donde la estimación supera el valor observado por un radar. Les mejora claramente el pronóstico pero aclaran que podría deberse en gran parte a los datos observados por radar y que deberían hacer una prueba en la que solo asimilen rayos.

\paragraph{Papadopoulos et al. 2009} \label{Papadopoulos et al. 2009}
\href{https://www.sciencedirect.com/science/article/pii/S0169809509001513}{\textbf{Evaluating the impact of lightning data assimilation on mesoscale model simulations of a flash flood inducing storm}}

Este trabajo utilizan la misma metodología que \nameref{Papadopoulos et al. 2005} y la evalúa en un caso de inundación repentina en Rumania. Asimilan los rayos de la red Arrival Time Difference (ATD, \textcolor{red}{Lee et al 1986}). Hacen una simulación de 72 horas en las que asimilan durante las primeras 60. Encuentran que asimilar los rayos les mejora el pronóstico y en especial les logra reproducir la tormenta que produjo la inundación, cosa que no se ve en la corrida control.

\paragraph{Pessi y Businger 2009} \label{Pessi y Businger 2009}
\href{https://journals.ametsoc.org/view/journals/mwre/137/10/2009mwr2765.1.xml}{\textbf{The Impact of Lightning Data Assimilation on a Winter Storm Simulation over the North Pacific Ocean}}

Asimilan rayos de manera similar a la realizada en \nameref{Pessi y Cherubini 2006} (transformando tasa de rayos a tasa de precipitación) para un caso de rápida intensificación de un ciclón extratropical que los pronósticos no vieron bien. Usan nuevamente la red PacNet y obtienen la relación con la precipitación a partir de comparar la tasa de rayos con la precipitación estimada por el TRMM. El esquema de asimilación es igual a los de \textcolor{red}{Manobianco et al. 1994}, \nameref{Alexander et al. 1999} y \nameref{Chang et al. 2001} pero con sutiles cambios. Hacen una corrida de 24 horas en las que hacen nudging en las primeras 8. La corrida que asimila les logra mejorar bastante los campos de superficie, presión y viento. Esto lo asocian a que la corrida control tiene mayor $T_{v}$ en toda la atmósfera asociado a mayor calor y humedad. Hacen también corridas de 36 y 48 horas arrancando inicializando 12 y 24 horas antes con resultados ambiguos. En el de 36 horas les aumenta el error en la ubicación y en el de 48 no se les profundiza lo suficiente, probablemente por la poca actividad eléctrica en los primeros plazos. Por último hacen un experimento de sensibilidad en el que ponen $\pm 1 \sigma$ a la estimación de precipitación los cuales son muy similares al original demostrando una insensibilidad a la relación entre rayos y precipitación, lo cual es beneficioso y coincide con lo encontrado por \nameref{Chang et al. 2001} indicando que importa más la ubicación de los rayos que la tasa de los mismos.

\paragraph{Fierro et al. 2012} \label{Fierro et al. 2012}
\href{https://journals.ametsoc.org/view/journals/mwre/140/8/mwr-d-11-00299.1.xml}{\textbf{Application of a Lightning Data Assimilation Technique in the WRF-ARW Model at Cloud-Resolving Scales for the Tornado Outbreak of 24 May 2011}}

Usan nudging para aumentar la humedad en la capa entre 0°C y -20°C si donde hay rayos la humedad es menor a un determinado valor para que en esas zonas aparezca convección. Usan datos de redes terrestres de rayos para simular datos del GLM. Usan dominios anidados con resoluciones que representan la resolución del GLM (9km), permitida (3km) y resuelta (1km). Los resultados para los dominios de 1 y 3km son similares. Usan WSM6. El proceso de asimilación es similar al de \nameref{Fierro y Reisner 2011}. Es el primer trabajo que asimila rayos en una resolución que permite no usar la parametrización de la convección.

\paragraph{Stefanescu et al 2013} \label{Stefanescu et al 2013}
\href{https://www.researchgate.net/publication/237082394_1D4D-VAR_data_assimilation_of_lightning_with_WRFDA_system_using_nonlinear_observation_operators}{\textbf{1D+4D-VAR data assimilation of lightning with WRFDA system using nonlinear observation operators}}

Usan la red ENTLN para generar un proxy de GLM para 2 casos de estudio. Para asimilar usan 3 formas de nDVAR transformando la tasa de rayos a CAPE. El primer esquema que plantean usa 3DVAR actualizando el rate de temperatura, los otros 2 se basan en 1D+nDVAR (n = 3, 4), a partir de 1DVAR obtienen retrievals de temperatura que luego los asimilan con el nDVAR. Para pasar de tasa de rayos a CAPE usan una relación entre la tasa de rayos y la máxima velocidad vertical y a su vez la relación entre esta y el CAPE. No suprimen la convección donde no se observan rayos. Acumulan rayos durante 20 minutos centrados en la hora del análisis. En el caso del 3DVAR asimilan directo los rayos al transformarlos a CAPE, en los 1D+nDVAR primero se aplica un 1DVAR para encontrar el perfil de temperatura más acorde a la tasa de rayos observada y ese perfil lo asimilan con el nDVAR. A estos perfiles 1D les aplican un chequeo de consistencia. Como modelo usan el WRF con 9 km de resolución y WSM6 en la microfísica. Los perfiles 1D generados tienden a calentar los niveles más bajos y a dejar estable/enfriar levemente los niveles medios y altos lo que aumenta la inestabilidad y el CAPE y por lo tanto los rayos en el modelo. Los resultados indican que el mejor esquema es 1D+4DVAR seguido por 1D+3DVAR y  por último 3DVAR.

\paragraph{Apodaca et al. 2014} \label{Apodaca et al. 2014}
\href{https://npg.copernicus.org/articles/21/1027/2014/}{\textbf{Development of a hybrid variational-ensemble data assimilation technique for observed lightning tested in a mesoscale model}}

Buscan evaluar el impacto de la asimilación de rayos en la gran escala y no solo en los procesos de mesoescala pensando principalmente en estudios climáticos. La metodología que se usa para asimilar es la Maximum Likelihood Ensemble Filter (MLEF). Las variables que buscan impactar con la asimilación son T, q, PD (hydrostatic pressure depth), U, V y CWM (masa de condensado total). Acumulan rayos durante 6 horas centradas en la hora del análisis en una retícula de 10 km. Como operador de las observaciones usan la metodología de \textcolor{red}{Price y Rind 1992} que relaciona la tasa de rayos con la velocidad vertical máxima. Encuentran que la expresión les sobrestima la tasa de rayos entonces aplican un factor de corrección que se adapta en cada ciclo. La simulación usa 2 dominio anidados, uno grande con resolución de 27 km y otro menor de 9 km. Hacen 3 simulaciones, una control, una que asimila todas las observaciones y una última que asimila solo una observación para analizar el impacto en las variables afectadas por la asimilación. El experimento de 1 observacion muestra que todas las variables presentan innovaciones no despreciables y a cierta distancia de la misma indicando la compleja estructura de la matriz de covarianzas del ensamble. Al comparar la corrida que asimila con la control se ve que los análisis presentan un menor error al comparar los rayos observados con los pronosticados pero esto no se observa en los pronósticos a 6 horas, probablemente porque asimilan solo rayos y estos no logran ajustar adecuadamente el entorno para que la convección se mantenga.

\paragraph{Fierro et al. 2014} \label{Fierro et al. 2014}
\href{https://journals.ametsoc.org/view/journals/mwre/142/1/mwr-d-13-00142.1.xml}{\textbf{Evaluation of a Cloud-Scale Lightning Data Assimilation Technique and a 3DVAR Method for the Analysis and Short-Term Forecast of the 29 June 2012 Derecho Event}}

En este trabajo comparan la metodología desarrollada por \nameref{Fierro et al. 2012} que usa nudging con 3DVAR asimilando radar. La evaluación la hacen para el caso de un derecho. El modelo que usan es el WRF con una resolución de 3 km. Los datos de rayos provienen de la ENTLN. La corrida en la que asimilan rayos le da mejores resultados que la que usan 3DVAR para asimilar radares porque en esta última le aparece convección espúrea que hace que el derecho simulado se encuentre siempre por delante de las observaciones. Realizan un par de corridas en las que sacan radares para ver cuál es el que al asimilarlo les genera la convección espúrea y al detectarlo los resultados son bastante similares a cuando asimilan solo rayos. Asimilando solo el radar problemático hacen simulaciones en las que asimilan solo reflectividad y viento radial, siendo este último el que genera los problemas (no encuentran el por qué).

\paragraph{Lagouvardos et al. 2014} \label{Lagouvardos et al. 2014}
\href{https://www.sciencedirect.com/science/article/pii/S0169809513001932}{\textbf{Study of a heavy precipitation event over southern France, in the frame of HYMEX project: Observational analysis and model results using assimilation of lightning}}

Buscan evaluar el impacto de la asimilación de rayos para un caso de precipitación intensa en Francia. Realizan una simulación de 72 horas para un dominio con resolución de 6 km con otro anidado de 2 km centrado en la región de mayor precipitación en donde resuelven explícitamente la convección. La metodología para asimilar los rayos es similar a la de \nameref{Mansell et al. 2007}. Asimilan durante las primeras 18 horas acumulando rayos durante 15 minutos. Les mejora la posición de las piletas de aire frío por tener mejor ubicada la convección y la posición y valores de los acumulados de precipitación.

\paragraph{Mansell 2014} \label{Mansell 2014}
\href{https://journals.ametsoc.org/view/journals/mwre/142/10/mwr-d-14-00061.1.xml}{\textbf{Storm-Scale Ensemble Kalman Filter Assimilation of Total Lightning Flash-Extent Data}}
Es uno de los primeros en tratar de asimilar rayos usando un filtro de Kalman por ensambles. Hacen un OSSE para evaluar el impacto de asimilar datos simulados de GLM. Para la microfísica utilizan una parametrización de 2 momentos y usan la parametrización de \textcolor{red}{Mansell et al. 2005} para la electrificación. Usando una relación lineal entre FED y volumen de graupel como operador de las observaciones no les difiere mucho de usar alguna relación conocida, especialmente después de varios ciclos de asimilación, pero si aumenta mucho el RMSE y el spread usar la parametrización de la actividad eléctrica. Tanto los movimientos verticales como la cantidad de graupel son bien estimadas al asimilar, especialmente usando la relación lineal para el caso de microfísica perfecta, pero el imperfecto tampoco anda muy mal. Destacan la dificultad que tiene el EnKF para los casos en que son pocos miembros los que ven convección, pero no logran solucionar el problema al duplicar la cantidad de miembros.

\paragraph{Marchand y Fuelberg 2014} \label{Marchand y Fuelberg 2014}
\href{https://journals.ametsoc.org/view/journals/mwre/142/12/mwr-d-14-00076.1.xml}{\textbf{Assimilation of Lightning Data Using a Nudging Method Involving Low-Level Warming}}

Plantean una metodología similar a la de \nameref{Fierro et al. 2012} pero en vez de aumentar la humedad para forzar la convección, aumentan la temperatura en la capa que genera las ascendentes. Para las simulaciones usan 3 dominios anidados de 27, 9 y 3 km asimilando solo en los últimos 2 duarante 12 horas después de 6 horas de spinup y dejan evolucionar por 12 horas más. En cada ciclo de asimilación toman los rayos ocurridos $\pm 5 $ minutos y solo les interesa si hubo actividad eléctrica, no cuánta. La metodología se basa asimilar usando nudging un perfil de temperatura adiabático entre el nivel de la parcela más inestable y su nivel de condensación por ascenso. A diferencia de \nameref{Fierro et al. 2012} usan un coeficiente de nudging menor para limitar la aparición de ondas acústicas en el modelo. El nudging lo realizan solo en los puntos donde el modelo no está pronosticando convección, lo cual lo definen en base al máximo graupel de la columna. Destacan algunos problemas asociados a esta metodología como que les produce tormentas más intensas que las observadas y que para algunos casos la innovación que quieren introducir con el nudging es 0 y por lo tanto no se modifican los perfiles. La metodología aplicada como la de \nameref{Fierro et al. 2012} que también prueban tienden a generar un exceso de precipitación en los primeros plazos de pronóstico y un déficit en las horas siguientes. No encuentran que alguno de las 2 metodologías sea mejor que la otra excepto para casos especiales.

\paragraph{Qie et al. 2014} \label{Qie et al. 2014}
\href{https://www.sciencedirect.com/science/article/pii/S016980951400194X}{\textbf{Application of total-lightning data assimilation in a mesoscale convective system based on the WRF model}}

Emplean una fórmula muy similar a la de \nameref{Fierro et al. 2012} para transormar la tasa de rayos a relación de mezcla de las especies microfísicas de la fase hielo. Emplean el modelo WRF con la parametrización WSM6 con un dominio cuya resolución es de 6 km con otro anidado de 2 km. Para estimar los parámetros de la fórmula emplean 3 simulaciones que matchearon bien a las observaciones a partir de las cuales compararon la relación de mezcla de cada especie con la tasa de rayos. Para asimilar corren el modelo normalmente y para los lugares en que hay rayos toman como relación de mezcla el máximo entre lo pronosticado y lo que da la fórmula, añaden también un parámetro para afectar a los puntos cercanos que no presentan actividad eléctrica. Hacen una simulación de 12 horas en las que dejan las 6 primeras de spin-up, asimilan durante las siguientes 2. Les mejora bastante tanto la comparación contra imágenes de radares como contra la percipitación acumulada, especialmente para los valores más altos.

\paragraph{Wang et al. 2014} \label{Wang et al. 2014}
\href{https://www.sciencedirect.com/science/article/pii/S0169809514002695}{\textbf{Improving forecasting of strong convection by assimilating cloud-to-ground lightning data using the physical initialization method}}

Usan una relación entre los rayos y un colmax de radar que luego transforman en un perfil de reflectividad y es lo que asimilan usando "inicialización física". Esta inicialización física consiste en generar el entorno convectivo en la posición donde se tienen rayos, a partir del proxy de radar obtenido se definen las concentraciones de las especies microfísicas y la velocidad vertical por ejemplo. Acumulan rayos durante 40 minutos (-30, +10). Realizan 5 experimentos, 1 de control y los otros 4 en los que van aumentando la cantidad de ciclos en 1 hora. Los experimentos que asimilan son claramente mejores que el control y el que hace 4 ciclos es el mejor pero los beneficios se pierden más allá de las 3 horas. Hacen un último experimento en el que hacen ciclos cada 3 horas con buenos resultados que se extienden hasta las 4 horas. El empeoramiento del pronóstico a las pocas horas lo relacionan a que solo actualizan unas pocas variables.

\paragraph{Fierro et al. 2015} \label{Fierro et al. 2015}
\href{https://journals.ametsoc.org/view/journals/mwre/143/3/mwr-d-14-00183.1.xml}{\textbf{Impact of Storm-Scale Lightning Data Assimilation on WRF-ARW Precipitation Forecasts during the 2013 Warm Season over the Contiguous United States}}

Utiliza la metodología de nudging desarrollada en \nameref{Fierro et al. 2012} para 67 casos. La corrida de control sin asimilación de rayos presenta un bias húmedo en la precipitación acumulada que es aumentado por la asimilación de rayos, especialmente en las primeras horas de pronóstico.

\paragraph{Lynn et al. 2015} \label{Lynn et al. 2015}
\href{https://journals.ametsoc.org/view/journals/wefo/30/2/waf-d-13-00028_1.xml}{\textbf{An Evaluation of the Efficacy of Using Observed Lightning to Improve Convective Lightning Forecasts}}

Misma metodología que \nameref{Fierro et al. 2012}, pero añaden que para zonas en que no hay rayos pero el modelo tiene convección hacen nudging para llevar los hidrometeoros a 0. El filtrado de la convección espúrea en el pronóstico les funciona bien.

\paragraph{Yang et al. 2015} \label{Yang et al. 2015}
\href{https://www.hindawi.com/journals/amete/2015/763919/}{\textbf{Assimilation of Chinese Doppler Radar and Lightning Data Using WRF-GSI: A Case Study of Mesoscale Convective System}}

Aplican la metodología desarrollada por \nameref{Weygandt et al. 2008} implementada en el GSI para un caso de estudio en China. Hacen 3 experimentos, 1 control y 2 en lo que asimilan en uno rayos transformado a radar y otro radar observado. Las corridas arrancan a las 00Z, de 06 a 09Z asimilan horariamente acumulando rayos durante 40 minutos (-30:10) y luego dejan evolucionar por 6 horas más. El tamaño de la reticula es $\approx$ 13 km. En general asimilar solo rayos da mejores resultados porque el radar tiende a generar incrementos mayores en las variables que generan un aumento excesivo de la convección y la precipitación que reducen los estadísticos respecto de la que usa rayos. En ambos casos la precipitación se encuentra sobrestimada. Los buenos resultados se mantienen por más de 6-7 horas. Detectan la problemática de que los rayos no se encuentran justamente en las zonas de mayor reflectividad entonces al asimilar la convección le queda desplazada respecto de la observada.

\paragraph{Allen et al. 2016} \label{Allen et al. 2016}
\href{https://journals.ametsoc.org/view/journals/mwre/144/9/mwr-d-16-0117.1.xml}{\textbf{Assimilation of Pseudo-GLM Data Using the Ensemble Kalman Filter}}

Hacen experimentos ideales a partir de sondeos de casos de convección multicelular y supercelular en base a lo ya hecho en \nameref{Mansell 2014}. Realizan primero una simulación en la que resuelven la electrificación de las tormentas para desarrollar modelos que pasen de las variables meteorológicas a rayos. Usan un filtro de Kalman por ensambles para asimilar los rayos usando los modelos generados. Los mejores resultados los encuentran usando una relación lineal (de lineal no tiene nada excepto para cantidades pequeñas de graupel) entre la tasa de rayos y el volumen o masa de graupel.

\paragraph{Dixon et al. 2016} \label{Dixon et al. 2016}
\href{https://www.researchgate.net/publication/305217287_The_Impact_of_Lightning_Data_Assimilation_on_Deterministic_and_Ensemble_Forecasts_of_Convective_Events}{\textbf{The Impact of Lightning Data Assimilation on Deterministic and Ensemble Forecasts of Convective Events}}

Evaluan el impacto de asimilar observaciones siguiendo la metodología de \nameref{Fierro et al. 2012} para pronósticos determinísticos y por ensambles usando observaciones de la red WWLLN en pronósticos de menos de 24 horas en un caso de un derecho. La metodología para asimilar difiere levemente respecto de \nameref{Fierro et al. 2012} en que con el nudging tienden a la saturación todos los niveles por debajo de los 200hPa en un entorno de 10 km de los rayos (cambiar estos valores no afectan sustancialmente los resultados), eliminan la dependencia con la microfísica empleada al no depender de la cantidad de graupel pronosticada. Acumulan rayos durante 5 minutos. Corren usando una resolución de 3 km y la microfísica es Thompson. En el caso del ensamble corren 64 miembros los cuales son generados a partir de perturbar el GFS en base a covarianzas encontradas entre los pronósticos de 24 y 48 horas a lo largo de un año. La corrida determinística que asimila logra pronosticar mucho mejor que la control el derecho hasta el final de la corrida pero con una deficiencia en la cantidad de precipitación estratiforme (lo asocian a que Thompson no es completamente de 2 momentos). Para el caso del ensamble previo a realizar el nudging hacen ciclos de EnKF asimilando otras observaciones cada 3 horas. El análisis de la corrida que no asimila rayos logra ver algo de la actividad pero no tan bien como al asimilarlos. Para los pronósticos la corrida que asimiló representa buenos resultados excepto en los últimos plazos donde no presenta la convección observada al igual que la control. No se ve una mejora tan grande como en el determinístico respecto de la corrida control.

\paragraph{Fierro et al. 2016} \label{Fierro et al. 2016}
\href{https://journals.ametsoc.org/view/journals/mwre/144/11/mwr-d-16-0053.1.xml}{\textbf{Assimilation of Flash Extent Data in the Variational Framework at Convection-Allowing Scales: Proof-of-Concept and Evaluation for the Short-Term Forecast of the 24 May 2011 Tornado Outbreak}}

El objetivo de este trabajo es evaluar el mismo caso que \nameref{Fierro et al. 2012} pero usando 3DVAR en vez de nudging para asimilar en conjunto con observaciones de radar. Los pronósticos los hacen usando el WRF con una resolución de 3 km. Hacen muchos experimentos en los que combinan radar y 2 redes de detección de rayos junto con varias opciones de configuración. En las ubicaciones en que se detectan rayos generan pseudo-observaciones de $q_{v}$ iguales al $q_{v}$ de saturación y eso es lo que ingresa al 3DVAR. \textcolor{red}{IMPORTANTE!!!!! en el paper se dice que donde no hay rayos no se hace nada, pensando en que el 3DVAR lo va a tomar como missing value pero un tiempo después hacen la corrección diciendo que en realidad estaba tomando una innovación de 0, haciendo que la información no se propague a zonas sin observaciones.} Hacen 1 solo ciclo de asimilación. Encuentran que asimilar la actividad eléctrica permite marcar mejor los zonas de mayores ascensos al combinarse con la asimilación de radar porque las primeras se concentran principalmente ahí y el radar actúa sobre un área más grande. En la revisión aclaran que los buenos resultados son porque no se propagaba la información por tener innovación 0 en donde no hay actividad eléctrica, pero que se logran reproducir esos resultados tomando una distancia de correlación menor para limitar la influencia. Es solo un caso el que analizan pero no se encuentra un claro bias positivo en la precipitación.

\paragraph{Giannaros et al. 2016} \label{Giannaros et al. 2016}
\href{https://www.sciencedirect.com/science/article/abs/pii/S1364815215301067}{\textbf{WRF-LTNGDA: A lightning data assimilation technique implemented in the WRF model for improving precipitation forecasts}}

Emplean metodologías similares a las descriptas por \nameref{Mansell et al. 2007} y \nameref{Lagouvardos et al. 2014} para 8 casos de precipitación en Grecia usando ventanas de asimilación más similares a las que se pueden usar en un pronóstico oeprativo. Para cada caso hacen 3 corridas, una control que no asimila, una que asimila y suprime la convección donde no se observan rayos y otra asimila pero no suprime. Cada corrida dura 30 horas en las que asimilan cada 10 minutos durante las primeras 6 horas. Hacen 2 tipos de distinciones entre los casos, aquellos que tuvieron mucha cantidad de observaciones de precipitación mayor a 20mm (WS) y los que no (NWS) y aquellos en que hubo muchos puntos de retícula con rayos (ILA) y los que no (SLA). Considerando todos los casos la corrida que no suprime la convección es la que brinda los mejores resultados. Los casos WS tienen mejor resultados en la control y aun así la asimilación mejora el pronóstico especialmente la corrida que no suprime, en los casos NWS se ve una mejora pero no tan clara como en los WS. Esto lo asocian a que el WRF suele presentar problemas al pronosticar estos casos y la asimilación parece que no logra corregir esto. Comparando los casos ILA y SLA no se ve una gran diferencia aunque en los ILA la mejora es mayor. Hacen un análisis detallado de uno de los casos utilizados.

\paragraph{Heath et al. 2016} \label{Heath et al. 2016}
\href{https://agupubs.onlinelibrary.wiley.com/doi/full/10.1002/2016MS000735}{\textbf{A simple lightning assimilation technique for improving retrospective WRF simulations}}

Este trabajo va en una línea distinta a todos los demás que en definitiva buscan generar una buena condición inicial para el un pronóstico, el objetivo de este es usar la metodología de \nameref{Mansell et al. 2007} para generar simulaciones retrospectivas con el WRF que tengan mejor representada la precipitación. Como este trabajo usa la versión actualizada de la parametrización de Kain-Fritsch para la convección y esta tiene la opción de generar shallow convection, agregan una nueva opción para suprimir la convección en los puntos donde no se tienen rayos en la que a la parametrización solo se le permite generar este tipo de convección. Hacen simulaciones para 3 meses diferentes, 2 julios y un enero como para ver casos en que la parametrización de la convección está muy y poco activa respectivamente. Hacen 4 corridas, una control y otras 3 en donde prueban las 3 formas distintas de suprimir la convección. La opción de supresión en la que permiten la shallow convection es la opción que mejor funciona. Suprimir completamente da buenos resultados al promediar pero al ver el día a día, les genera excesivos acumulados de precipitación. Las mejoras con la asimilación también se ven en variables como T2 y las mejoras también son en la que permite shallow convection. En la simulación de enero al haber menos actividad eléctrica el pronóstico es levemente mejor pero no hay evidencia de que empeore el pronóstico implicando que se podría aplicar esta metodología de manera anual.

\paragraph{Federico et al. 2017a} \label{Federico et al. 2017a}
\href{https://nhess.copernicus.org/articles/17/61/2017/}{\textbf{Improvement of RAMS precipitation forecast at the short-range through lightning data assimilation}}

Emplean una metodología similar a la de \nameref{Fierro et al. 2012} con leves modificaciones en el modelo RAMS a 20 casos de estudio en Italia. Usan observaciones de la red LINET y acumulan cada 5 minutos hacia atrás. Los cambios que hacen respecto de \nameref{Fierro et al. 2012} es que cambian el valor de unas constantes para ajustarse a las diferencias de resolución y tiempo en que acumulan rayos. Además no hacen nudging del vapor de agua sino que reemplazan el valor que les da la estimación por el pronósticado si está por debajo de cierto umbral. Realizan 3 corridas, una control, una que asimila durante 6 horas y pronostica las próximas 3 (de esta hacen 8 corridas lanzadas cada 3 horas) y una última en la que asimilan siempre y no pronostican. Considerando la precipitación acumulada durante un dia (suman las 3 horas de pronóstico de las 8 corridas que asimilan) se ve que la precipitación se encuentra aumentada en casi todo el dominio respecto del control excepto en algunos lugares específicos asociados a precipitaciones que se posicionaron levemente diferente. Destacan que esta metodología no tiende a empeorar respecto del control al evaluar la cantidad de estaciones en las que la precipitación observada y pronosticada están por encima de cierto umbral. Al analizar los scores para los 20 casos encuentran que para los umbrales más altos está muy sobrestimada la frecuencia, incluso en la corrida control (asociado al uso del modelo RAMS con WSM6) y se encuentra aumentado en las corridas que asimilan. Al aumentar la precipitación la asimilación los estadísticos le mejoran pero también le aumentan las falsas alarmas.

\paragraph{Federico et al. 2017b} \label{Federico et al. 2017b}
href{https://asr.copernicus.org/articles/14/187/2017/}{\textbf{Impact of the assimilation of lightning data on the precipitation forecast at different forecast ranges}}

Es un estudio muy similar al del \nameref{Federico et al. 2017a} en el que usan los mismos 20 casos y la misma metodología (el único cambio que hacen es que añaden vapor de agua en los 4 puntos adyacentes al que tiene rayos) pero en este caso realizan pronósticos a distintos rangos apra evaluar cuanto dura el impacto de la asimilación. Para evaluar el impacto a diferentes plazos siguen la misma estrategia que en \nameref{Federico et al. 2017a}, una corrida de control, una que asimila siempre y varias corridas que asimilan durante 6 horas y luego pronostican a 3, 6, 12 y 24 horas. Claramente los pronósticos a 3 horas acumulados para formar el de 24 horas tienen los mejores resultados y el impacto va decreciendo a medida que se incremente el plazo de pronóstico pero siempre mejor que el control.

\paragraph{Wang, H et al. 2017} \label{Wang, H et al. 2017}
\href{https://agupubs.onlinelibrary.wiley.com/doi/full/10.1002/2017JD027340}{\textbf{Improving Lightning and Precipitation Prediction of Severe Convection Using Lightning Data Assimilation With NCAR WRF-RTFDDA}}

Proponen una metología basada en estimar el contenido de graupel a partir de observaciones de rayos y lo asimilan en conjunto con el calor latente liberado (por la adición de graupel) via nudging. Acumulan rayos cada 15 minutos. La relación que usan entre los rayos y el contenido de graupel (integrado entre las isotermas de -10 y -40 °C) es lineal derivada por \textcolor{red}{Carey et al. 2014} y algunos estudios dicen que podría aplicar para cualquier punto del globo, no como la relación con la precipitación que es local. Como el valor de graupel que da la relación es integrado, a partir de otras simulaciones generan una relación entre esta variable y sus perfiles verticales que son los que van a terminar asimilando. Generan también graupel en los puntos adyacentes a los rayos usando pesos del tipo Cressman, aplican una corrección para mantener la cantidad de graupel en todo el dominio despues de este proceso. Hacen 3 experimentos uno en el que inicial la simulación al mismo tiempo que la asimilación (en este experimento también corren la metodología de \nameref{Fierro et al. 2012} para comparar), otro en que le dan unas horas de spin-up al modelo antes de empezar a asimilar y otro en el que van cambiando el tiempo que dura la asimilación y evalúan el impacto en los pronósticos a 2 horas. En los experimentos la asimilación del graupel y el calor latente brinda mejores resultados que la de \nameref{Fierro et al. 2012} debido a que genera precipitación más rápido y los campos de radar pro ejemplo son más parecidos a los observados. En uno de los casos que analizan el entorno pronosticado está bastante más seco que el observado y aunque el análisis generado es bueno la convección no logra perdurar en el pronóstico ya que esa componente no es algo que puede arreglar con la asimilación empleada.

\paragraph{Wang, Y et al. 2017} \label{Wang, Y et al. 2017}
\href{https://www.mdpi.com/2073-4433/8/3/55}{\textbf{A Case Study of Assimilating Lightning-Proxy Relative Humidity with WRF-3DVAR}}

Usan la transformación de \nameref{Fierro et al. 2012} para pasar de rayos a vapor de agua para generar pseudo-observaciones que asimilan usando 3DVAR. Hacen 4 experimentos, uno control que asimila solo datos convencionales, uno que asimila radar (reflectividad y viento radial), uno que asimila rayos y un último que asimila radar y rayos. Los campos de temperatura y viento responden bien a la asimilación pero los campos de reflectivad se ven ampliamente sobrestimados. La precipitación también esta super sobrestimada, incluso el FSS del control supera algunos experimentos para algunos thresolds. Los autores dicen que el pronóstico en su conjunto mejora pero es muy polémico.

\paragraph{Zhang et al. 2017} \label{Zhang et al. 2017}
\href{https://link.springer.com/article/10.1007/s13351-017-6133-3}{\textbf{Assimilation of Total Lightning Data Using the Three-Dimensional Variational Method at Convection-Allowing Resolution}}

Generan una relación entre la tasa de rayos y la humedad relativa para generar pseudo observaciones que asimilan en ciclos de 10 minutos de 3DVAR (con observaciones centradas). Estas pseudo observaciones las generan con la metodología de \refname{Fierro et al. 2012} haciendo la transformación de $q_{v}$ a HR porque es la variable de humedad que maneja el WRFDA. Realizan 5 corridas, 1 control, 1 en la que asimilan 1 sola vez y otras 3 en las que van cambiando la duración del período en que asimilan (30, 60 y 90 minutos). Las corridas que asimilan mejoran claramente respecto del control excepto la que hace un solo ciclo de asimilación. Cuantos más ciclos corren mejores son los resultados pero no se ve una mejora significativa más allá de los 60 minutos. Lo asocian a que los incrementos que da el 3DVAR van cayendo con los ciclos entonces para los últimos los incrementos son pequeños.

\paragraph{Chen et al. 2019} \label{Chen et al. 2019}
\href{https://www.sciencedirect.com/science/article/pii/S0169809518308585}{\textbf{Lightning data assimilation with comprehensively nudging water contents at cloud-resolving scale using WRF model}}

Hacen una combinación de las metodologías de \nameref{Fierro et al. 2012} y \nameref{Qie et al. 2014}, usan la primera para asimilar vapor de agua en los niveles bajos y la segunda para el graupel en la región de fase mixta. Para definir las constantes en las expresiones que se usan para el nudging definen umbrales en base al Bul Richardson Number. Hacen 4 corridas, una control y 3 que asimilan, usando los esquemas citados y el que los combina. Acumulan rayos cada 10 minutos. Evalúan en 2 casos de líneas de instabilidad. Asimilan durante 3 horas y pronostican por las siguientes 5. Aunque el control logra reproducir bien las tormentas del primer caso, los experimentos que asimilan dan un mejor resultado, siendo el mejor el que combina las metodologías porque con la asimilación de vapor en niveles bajos logra incrementar las ascendentes que contrarestan las descendentes de la asimilación de graupel en niveles medios y por lo tanto los topes son más altos. Las piletas de aire frío se encuentran presentes en todas la simulaciones pero la que más se asemeja a las observaciones es la del que combina las metodologías, la de Fierro es muy débil y la de Qie muy intensa. Esto se propaga a los pronósticos de PP y radar, con Qie por ejemplo las tormentas decaen demasiado rápido y no perduran en el tiempo mientras que en el que combina ambos campos son semjantes a los observados.

\paragraph{Federico et al. 2019} \label{Federico et al. 2019}
\href{https://nhess.copernicus.org/articles/19/1839/2019/}{\textbf{The impact of lightning and radar reflectivity factor data assimilation on the very short-term rainfall forecasts of RAMS@ISAC: application to two case studies in Italy}}

Asimilan en conjunto radar y rayos para 2 casos de convección que no fue bien pronosticada en Italia. Para asimilar rayos usan la metodología de \nameref{Fierro et al. 2012} y para el radar a partir de la reflectividad obtienen perfiles de humedad relativa que asimilan con 3DVAR. En los 2 casos asimilan en un dominio con resolución de 4 km y en uno de ellos anidadan uno con una resolucion de aproximadamente 1 km pero no asimilan. Asimilan durante 6 horas y pronostican por 3. La ventana que usan para los datos de rayos es de 5 minutos hacia atras. Para asimilar radar pasan los perfiles del modelo a perfiles de reflectividad y a partir de las diferencias con las observaciones obtienen pesos que usan para generar perfiles de humedad relativa, estos perfiles los asimilan con 3DVAR. El 3DVAR solo ajusta la humedad relativa, no tiene correlaciones con otras. Hacen 4 experimentos, uno control, uno que asimila radar, otro rayos y el último ambos. Para el primer caso analizado (menos convectivo), los experimentos que asimilan mejoran mucho respecto del control pero el que mejor resultados da es el que asimila ambas fuentes. Para el caso más convectivo analizan 2 períodos de 3 horas en ambos el experimento que asimila ambas fuentes siempre es el mejor y los que asimilan 1 sola son uno mejor en un período y otro en el otro.

Tienen un paper suplementario en el que analizan más cosas como la cantidad de PP e hidrometeoros que agregó el nudging, la sensibilidad a los parámetros de la ecuación que transforma rayos a $q_{v}$, la sensibilidad al error en la observación de radar, sensibilidad a las condiciones iniciales y de borde y sensibilidad a la cantidad de niveles verticales en el modelo.

\paragraph{Fierro et al. 2019} \label{Fierro et al. 2019}
\href{https://journals.ametsoc.org/view/journals/mwre/147/11/mwr-d-18-0421.1.xml}{\textbf{Variational Assimilation of Radar Data and GLM Lightning-Derived Water Vapor for the Short-Term Forecasts of \\ High-Impact Convective Events}}

ESte trabajo es una continuación de \nameref{Fierro et al. 2016} y tratan de atacar 2 problematicas de la metodología, 1) aparentemente sigue habiendo un bias positivo en la precipitación y 2) el agregado de vapor de agua no es balanceado con una extracción igual en otras zonas del dominio. Usan datos medidos por el GLM. Hacen varios experimentos asimilando rayos y radar solos o combinados, uno en el que imponen la conservación del vapor de agua para que lo que se agrega se quite en otro lado y uno en el que usan varios ciclos cortos de asimilación en vez de uno de 1 hora como para los experimentos anteriores. En los resultados encuentran que los mejores pronósticos se dan cuando se asimila GLM. Al usar scores espaciales se ve que al comparar contra observaciones de radar usar solo GLM es lo mejor pero al hacerlo contra la precipitación los mejores resultados se obtienen al combinar GLM y radar. Al asimilar en ciclos de 10 minutos usar solo GLM intensifica demasiado las tormentas en el pronóstico. Plantear la conservación del vapor de agua no afectó casi en nada los resultados.

\paragraph{Chen Y. et al. 2020} \label{Chen Y. et al. 2020}
\href{https://www.mdpi.com/2072-4292/12/7/1165}{\textbf{Case Study of a Retrieval Method of 3D Proxy Reflectivity from FY-4A Lightning Data and Its Impact on the Assimilation and Forecasting for Severe Rainfall Storms}}

Usan las mediciones de rayos del satélite FY-4A (que tiene montado el GLM chino) para transformar de rayos a radar y asimilarlas comparando con los resultados de asimilar radar directamente. Usan el producto flash que provee el sensor. Para pasar de rayos a radar primero calculan la densidad de rayos acumulando en una retícula regular de 0.1° durante 1 hora (-20:+40 de la hora de análisis) para luego convertir a colmax y después a un volumen. Para pasar de densidad de rayos a colmax usan una relación logarítmica (la comparan con las relaciones incluidas en el GSI y estas están subestimadas.) y para pasar al volumen calcularon estadísticamente perfiles de radar a partir de los colmax y asimilan los datos entre 1000 y 10000m cada 1000. Además de los rayos, asimilan datos convecionales, reflectividad y viento radial en ciclos de 3 horas que a partir de estos hacen ciclos de 3 ciclos de 1 hora y pronostican por 6. Hacen 4 experimentos, el control que solo asimila datos convencionales, uno que suma viento radial, otro que añade reflectividad y un último que en vez de la reflectividad de radar asimila el proxy de reflectividad de los rayos. Analizando un caso altamente precipitante, las corridas que asimilan tienen un mejor desempeño que la control pero la que incluye rayos no logra alcanzar la performance de la que solo usa radar, lo asocian al lag temporal entre rayos y reflectividad.

Para analizar una de las ventajas que tienen los rayos que es su cobertura en regiones de terreno complejo en donde los radares tienen dificultades analizan también un caso de precipitación intensa en una región montañosa. Hacen 3 experimentos, uno control que no asimila, uno que asimila datos convencionales y otro que agrega los rayos. En este caso se ve también una mejora en los campos de preciptación al asimilar rayos, es el único que logra pronosticar relativamente bien la ubicación de la lluvia.

\paragraph{Chen Z. et al. 2020} \label{Chen Z. et al. 2020}
\href{https://agupubs.onlinelibrary.wiley.com/doi/abs/10.1029/2020JD033330?af=R&sid=researcher&utm_campaign=RESR_MRKT_Researcher_inbound&utm_medium=referral&utm_source=researcher_app}{\textbf{A method to update model kinematic states by assimilating satellite-observed total lightning data to improve convective analysis and forecasting}}

Asimilan las observaciones de rayos del LMI a partir de transformarlos a $w_{max}$ usando la relación de \textcolor{red}{Price y Rind et al. 1992} usando 3DVAR. Usan los eventon es lugar de grupos y flashes para tener toda la información posible y evitar propagar errores asociados a los algoritmos de clusterización. Aplican un control a los datos del LMI, este consiste en: 1) eliminar los eventos aislados, generalmente son ruido, 2) agrupan eventos cada 15 minutos, 3) agrupan segun la retícula del modelo. A partir de la CDF de tasa de rayos obtuvieron 3 valores a partir de los cuales transformarlos a velocidad vertical máxima definidos por los percentiles 60, 80 y 90. A partir de la velocidad vertical máxima obtenida generan un perfil de $w$ que luego transfoman a convergencia horizontal via ecuación de continuidad porque asimilar $w$ puede generar ruido. La relación entre $w_{max}$ y el perfil lo obtienen estadísticamente a partir de varios casos. Hacen experimentos en que asimilan una sola observación en los que asimilan o una observación de viento radial de radar, una sola de rayos y una combinación de ambas. La que combina las fuentes es la que mejor da. Hacen 4 experimentos, uno control que solo asimilia datos convencionales, uno que añade radar, otro que añade rayos y un último que añade los dos. Las corridas que asimilan claramente superan la control, seguidas por la de rayos, la de radar y la que combina que dio los mejores resultados. La de rayos funciona mejor para los valores más altos de precipitación porque están más asociados a la actividad eléctrica. En los análisis se ve que el que combina genera mayores convergencias y tiene más humedad en niveles medios porque aunque la humedad solo es actualizada por las observaciones de radar, las mayores velocidades verticales que generan las observaciones de rayos redistribuyen el vapor de agua de niveles bajos. Este cambio en la humedad genera CAPE's mayores también.

\paragraph{Comellas Prat et al. 2020} \label{Comellas Prat et al. 2020}
\href{https://www.sciencedirect.com/science/article/pii/S0169809520311832}{\textbf{Lightning data assimilation in the WRF-ARW model for short-term rainfall forecasts of three severe storm cases in Italy}}

Utilizan la metodología desarrollada por \nameref{Fierro et al. 2012} para 3 casos que representan regímenes diferentes de precipitación en Italia. Usan el modelo WRF con una resolución de 3 km, a pesar de tener una resolución alta emplean una parametrización para la convección. Estos 3 casos los inicializan tanto usando GFS como ECMWF. Para parametrizar la microfísica emplean Thompson. Detectan que las mejoras en la precipitación se dan principalmente para los casos asociados a MCSs.

\paragraph{Hu et al. 2020} \label{Hu et al. 2020}
\href{https://journals.ametsoc.org/view/journals/mwre/148/3/mwr-d-19-0198.1.xml}{\textbf{Exploring the Assimilation of GLM-Derived Water Vapor Mass in a Cycled 3DVAR Framework for the Short-Term Forecasts of High-Impact Convective Events}}

Usan 3DVAR para asimilar pseudo observaciones de humedad específica en donde hay tasa de rayo mayor a 0. En esos puntos generan observaciones de 95\% de humedad relativa. Usan datos reales del GLM. Hacen varias pruebas de sensibilidad. Cantidad de ciclos de asimilación (cada 15 minutos o 1 hora), cantidad de tiempo que colectan rayos y escala de decorrelación de las observaciones. Los mejores resultados los tienen asimilando cada 15 minutos, colectando rayo de los últimos 10 minutos y usando una longitud de decorrelación de 3km (la simulación la hacen con una resolución de 1.5km). Con esta mejor configuración hacen pruebas asimilando rayos, radar (VAD, viento radial y reflectividad) y radar + rayos. Los mejores resultados los tienen asimilando radar + rayos pero es muy sutil.

\paragraph{Liu et al. 2020} \label{Liu et al. 2020}
\href{https://www.frontiersin.org/articles/10.3389/feart.2020.00288/full}{\textbf{An Approach for Assimilating FY4 Lightning and Cloud Top Height Data Using 3DVAR}}

Este trabajo aborda el problema de entre qué alturas generar las pseudoobservaciones de vapor de agua para asimilar los rayos, evalúan la de asimilar entre 2 isotermas, entre 2 alturas fijas y proponen una nueva en la que toman la capa comprendida entre el LCL y la altura del tope de nubes estimada por el mismo satélite que tiene el detector de rayos. Agrupan rayos durante 1 hora centrada en la del análisis. Las pseudo observaciones las vapor de agua las generan entre el LCL del modelo y el CTH (cloud top height) del satélite y tienen un valor de 90 \% y se aplican solo donde el modelo está por debajo de este porcentaje y las asimilan usando 3DVAR. Hacen 4 experimentos en 2 casos de estudio, uno control en el que no se asimila, 1 en el que generan pseudo observaciones entre las isotermas de 0 y 20 °C (LDA\_T020) siguiendo a \nameref{Fierro et al. 2012}, otro en el que las hacen entre el LCL y 15 km (LDA\_15km) siguiendo a \nameref{Fierro et al. 2016} y un último en que usan entre el LCL y el CTH (LDA\_CTH). A la hora del análisis los incrementos en el vapor de agua son mínimos para LDA\_T020 y máximos para LDA\_15km, esto se explica por el espesor de atmósfera involucrado, el LCL se encuentra por debajo de la isoterma de 0 °C y el CTH generalmente por debajo de los 15 km. En los pronósticos el LDA\_T020 es tan parecido al control que ni lo analizan, los otros dos incrementan claramente la convección pero el LDA\_15km genera ascensos y convergencias más intensas. En los campos de tempertura se ve como las corridas que asimilan tienen menor temperatura cerca de superficie asociado a la cold pool y mayor en niveles medios y altos en las zonas de convección. En los hidrometeoros se ven incrementos en las regiones de convección pero son mayores en LDA\_15km por introducir más vapor de agua. Al comparar con la precipitación, LDA\_15km al añadir más vapor de agua activa la convección más rápido y tienen un ETS más alto para los primeros plazos pero LDA\_CTH tiende a ser mejor.

\paragraph{Kong et al. 2020} \label{Kong et al. 2020}
\href{https://journals.ametsoc.org/view/journals/mwre/148/5/mwr-d-19-0192.1.xml}{\textbf{Assimilation of GOES-R Geostationary Lightning Mapper Flash Extent Density Data in GSI EnKF for the Analysis and Short-Term Forecast of a Mesoscale Convective System}}

Utilizan las relaciones lineales que definió el \nameref{Allen et al. 2016} para un caso real usando el filtro de Kalman por ensambles y hacen un estudio para ajustarlas porque se habían estimado en simulaciones ideales. Emplean datos reales del GLM. Analizan las innovaciones y covarianzas entre la tasa de rayos y variables como la velocidad vertical y la cantidad de hidrometeoros. Por último hacen un estudio para en el que la tasa de rayos asimilada sólo afecta a variables específicas (Solo la cantidad de graupel, todas menos la humedad específica y la temperatura y todas menos la velocidad vertical) y los mejores resultados son usando todas y todas menos la velocidad vertical.

\paragraph{Torcasio et al. 2020} \label{Torcasio et al. 2020}
\href{https://www.mdpi.com/2073-4433/11/5/541}{\textbf{Application of Lightning Data Assimilation for the 10 October 2018 Case Study over Sardinia}}

En este trabajo usan la metodología de \nameref{Fierro et al. 2012} para un caso altamente precipitante (>300 mm/24hs) que el pronóstico no vio. Evalúan 2 períodos de actividad, el primero no lo vio el modelo pero el segundo si, entonces buscan analizar como mejorar la predicción del primero con la asimilación afecta al segundo. Realizan 4 experimentos, 1 control que no asimila, 1 con una reoslución de 3 km y haciendo pronósticos a 3 horas, uno en el que hacen pronósticos a 1 hora y un último que hace pronósticos a 1 hora pero con una resolución de 2 km. Para el primer período de actividad analizado aumentar la frecuencia de los análisis mejora notablemente los pronósticos debido a que el modelo no puede generar/sostener la actividad convectiva. En los scores de los radares simulados respecto de la observación se ve hacer ciclos de 1 hora es mejor que de 3 y aumentar la resolución mejora los resultados. Para el segundo período analizado se repiten los mismos resultados aunque el control anduvo mejor.

\paragraph{Vendrasco et al. 2020} \label{Vendrasco et al. 2020}
\href{https://www.sciencedirect.com/science/article/pii/S0169809519308622}{\textbf{Potential use of the GLM for nowcasting and data assimilation}}

Arman una relación entre los rayos y las especies microfísicas y perfiles de radar para evaluar su uso en la asimilación de datos. A partir de una red de superficie generan un proxy de GLM (el período de estudio es pre-GLM por eso arman un proxy) el cual dividen en 6 clases a partir de la densidad de rayos y tomando la media de los perfiles asociados a cada rayo de cada clase obtuvieron la relación entre rayos y radar. De manera similar obtienen los perfiles de procentaje de cada tipo de hidrometeoro para cada clase de densidad de rayo. Para asimilar los perfiles de reflectividad usan 3DVAR. Hacen un análisis de cómo varían las variables polarimétricas para cada clase de densidad de rayos. Hacen 3 experimentos, uno control que no asimila, uno que asimila reflectividad y doppler y un último que asimila la reflectividad generada por el proxy de GLM. Hacen 4 ciclos de asimilación y luego pronostican por 12 horas. En general notan que la asimilación de los rayos mejora el pronóstico pero no llega a la performance de la asimilación de radar, principalmente porque esta asimila también zonas estratiformes mientras que la de rayos solo en zonas convectivas y porque incluye los datos de viento también.

\paragraph{Wang et al. 2020} \label{Wang et al. 2020}
\href{https://www.sciencedirect.com/science/article/pii/S0169809519315352}{\textbf{An improvement of convective precipitation nowcasting through lightning data dynamic nudging in a cloud-resolving scale forecasting system}}

Usan la relación propuesta por \textcolor{red}{Price y Rind 1992} para pasar de rayos a velocidad vertical máxima y asimilar perfiles de velocidad vertical via nudging. Para pasar de la velocidad vertical máxima a un perfil usan una relación obtenida a partir de 1 mes de corridas de un modelo. Hacen 5 experimentos en un caso de estudio, uno que no asimila y los otros 4 en los que va aumentando progresivamente la cantidad de plazos en que aplican el nudging. Encuentran que asimilar más veces no necesariamente se traduce en un mejor pronóstico. La velocidad vertical no ve particularmente afectada entre los distintos experimentos pero si la extensión horizontal de los valores, a más ciclos mayor es el área. Hacen un par de experimentos más en los que suman radar de manera sincrónica (asimilan al mismo tiempo que los rayos) y asincrónica (asimilan en tiempos diferentes que los rayos) y detectan que ambos casos mejoran el pronóstico, en especial para los umbrales chicos de PP que los rayos no ven por estar asociados a convección, pero el asincrónico tiene un mejor desempeño. El último análisis que hacen es correr el experimento asincrónico y el que asimila rayos solo veces (era el de mejor resultados) durante 4 meses. Ambos experimentos andan mejor que el control pero el asincrónico da mejor, especialmente para los umbrales bajos.

\paragraph{Zhang et el. 2020} \label{Zhang et al. 2020}
\href{https://link.springer.com/article/10.1007/s13351-020-9145-3}{\textbf{Application of Lightning Data Assimilation to Numerical Forecast of Super Typhoon Haiyan (2013)}}

Usan la metodología desarrollada por \nameref{Fierro y Reisner 2011} para asimilar datos de rayos medidos con la WWLLN en un huracán. Aplican la aproximación que usa \textcolor{red}{Dixon et al. 2016} en la que donde hay rayos llevan la humedad al 90\% entre superficie y 200hPa. Generan pseudo-observaciones de sondeo usando el formato LITTLE\_R del WRFDA asimilando cada 1 hora usando 3DVAR. Corren con 3 dominio anidados donde el de mayor resolución tiene 3km, en todos usan WSM6 y excepto en el de 3 km parametrizan la convección. Solo asimilan en el dominio de 3 km. Llegan a la conclusión de que asimilar los rayos que se dan más cerca del ojo (menos de 100 km) tiene los impactos más positivos y estos se pueden ver en pronósticos de hasta 48 horas.

\paragraph{Federico et al. 2021} \label{Federico et al. 2021}
\href{https://www.mdpi.com/2073-4433/12/8/958}{\textbf{Impact of Radar Reflectivity and Lightning Data Assimilation on the Rainfall Forecast and Predictability of a Summer Convective Thunderstorm in Southern Italy}}

Asimilan radar y rayos para un evento de precipitación intensa muy localizada en Palermo ($120 \frac{mm}{3hs}$). Hacen 7 experimentos, uno control, 2 en que asimilan solo radar o rayos y 4 en que asimilan ambos y estos se diferencian en la cantidad de ciclos que hacen para ver la predictibilidad del evento. Arrancan las corridas 8 horas antes del inicio del evento y durante las cuales asimilan en ciclos de 30 minutos y pronostican por las siguientes 3 horas (excepto en los experimentos en que asimilan por menos ciclos en los que van sacando de a uno). Los rayos los asimilan a partir de generar perfiles de $q_{v}$ correspondientes a 100 \% de humedad relativa que se ingresan a un 3DVAR. A la hora del último análisis (la hora en que arranca el evento) el único experimento que ve la actividad es el que asimila rayos ya que el radar que cubre esa zona justo estaba apagado por mantenimiento. Viendo la precipitación pronosticada para el evento, el control no ve nada para la zona pero sí para regiones cercanas, la que asimila radar logra ver algo mínimo pero bien ubicado porque toma algo de información de radares cercanos, la que asimila solo rayos lo ubica perfecto pero con la mitad de la precipitación observada y la que mejora anda es la que los combina que lo ubicó perfecto también y con una intensidad levemente menor. El asimilar radar en conjunto con los rayos hace que el pronóstico tenga un poco más de humedad en la zona de Palermo y la convección se mantenga por un período más prolongado. Para estudiar la predictibilidad usan los experimentos que dejaban de asimilar antes. El evento pudo ser relativamente bien pronosticado (con errores en intensidad y localización) incluso con análisis de 2 horas previas al evento.

\paragraph{Gan et al. 2021a} \label{Gan et al. 2021a}
\href{https://link.springer.com/article/10.1007/s13351-021-0092-4}{\textbf{Assimilation of Radar and Cloud-to-Ground Lightning Data Using WRF-3DVar Combined with the Physical Initialization Method—A Case Study of a Mesoscale Convective System}}

Asimilan perfiles de humedad relativa obtenida de mediciones de radar y rayos a través de una inicialización física que luego ingestan con 3DVAR. Usan los datos de radar para obtener los perfiles de regiones donde la precipitación estimada es mayor a 0.1 $\frac{mm}{hr}$ y los rayos para limitar aún más estas regiones al pedir que tenga que haber actividad eléctrica también. Realizan una corrida control y 5 que asimilan, 2 usan la inicialización física y se diferencias porque uno pone el constraint de la actividad eléctrica y el otro no y los otros 3 usan 3DVAR para asimilar los perfiles de la inicialización física. 2 de ellos asimilan solo radar y cambian el error de la observación y el otro pone el constarint de la actividad eléctrica. Las corridas hacen un spin-up de 12 horas, asimilan 3 y pronostican 9. A la hora del último análisis los incrementos de los experimentos que no usan los rayos son sustancialmente menores por la menor cantidad de datos asimilados. Viendo los pronósticos de reflectividad, los experimentos que usan los rayos se desempeñan mejor que los que no y los que usan 3DVAR andan mejor que los de inicialización física. Los campos de precipitación también están mejor representados por los experimentos que usan los rayos, el control subestima y los que no los usan sobrestiman. Remarcan que esta metodología es solo aplicable para casos convectivos por la poca/nula actividad eléctrica en precipitaciones estratiformes.

\paragraph{Gan et al. 2021b} \label{Gan et al. 2021b}
\href{https://agupubs.onlinelibrary.wiley.com/doi/10.1029/2020JD034300}{\textbf{Assimilation of the Maximum Vertical Velocity Converted From Total Lightning Data Through the EnSRF Method}}

Asimilan los rayos a partir de una relación entre la tasa de rayos y la velocidad vertical máxima usando un Ensemble Square Root Filter (EnSRF). Lo aplican para 2 casos donde en 1 hacen un análisis más exhaustivo. En el primero hacen 4 experimentos, uno control y 3 en donde asimilan, 1 en donde no usan un localización vertical, otro en el que asimilan hasta los -5 °C y un último en el que usan la localización vertical "Global Group Filter" (GGF). Este último y el control son los que corren para el segundo caso. En ambos casos hacen 9 horas de spin-up, 3 de asimilación haciendo ciclos de 1 horas. A la hora del análisis el error en el pronóstico es claramente reducido en los experimentos que asimilan, el que no localiza es el que presenta mejor RMSE de $W_{max}$ para los puntos con rayos y el que localiza con GGF es el que añade más hidrometeoros y por lo tanto presenta reflectividades más altas. Analizando los pronósticos, los experimentos que asimilan superan ampliamente al control, especialmente en las primeras horas. La reflectividad tiene un gran desempeño en las primeras horas y luego tienden a subestimar con el mejor experimento siendo el que usa la localización GGF. En cuanto a la precipitación, para el umbral de 1 mm todos los plazos sobrestiman la frecuencia pero para el de 5 mm esto se da solo para los plazos cortos al igual que en la reflectividad. A modo global el experimento que localiza con GGF resulta ser el mejor.

\paragraph{Honda et al. 2021} \label{Honda et al. 2021}
\href{https://agupubs.onlinelibrary.wiley.com/doi/full/10.1029/2021JD034611}{\textbf{Potential Impacts of Lightning Flash Observations on Numerical Weather Prediction With Explicit Lightning Processes}}

Hacen un experimento idealizado en el que asimilan observaciones del GLM resolviendo explícitamente la convección y usando el LETKF. Hacen la comparación entre asimilar rayos y radianzas en un OSSE. Para inicializar la convección, en la nature run usan una burbuja cálida en un entorno inestable y las perturbaciones del ensamble a partir de diferentes temperaturas en de la burbuja, su ubicación y el perfil de vientos del entorno. Hacen 4 experimentos, uno control, uno que asimila radar cada 5 minutos, uno que se ejecuta desprende del que asimila radar despues de 25 minutos y que asimila al final del pronóstico los datos sintéticos de GLM y un último igual al anterior pero que no asimila. Hacen un análisis de las covarianzas entre observaciones de radar, IR y GLM para ver el impacto de asimilarlas en el contenido de hidrometeoros. Claramente el radar tiene un alto impacto, el IR es muy reducido por la presencia del yunque y GLM tiene un impacto intermedio pero más cercano al del radar incluso para niveles bajos indicando que aunque sea un campo 2D tiene influencia en toda la columna. GLM presenta buena correlación con otras variables como la velocidad vertical y temperatura. Algo a destacar es que la máxima correlación con la velocidad vertical se encuentra desplazada respecto de le observacion. Viendo la correlación con cada hidrometeoro se ve que la mayor se da con el graupel. Otra cosa que analizan son las correlaciones en función de la cantidad de tiempo que usan para colectar los rayos, a mayor tiempo mayores son las correlaciones pero esto puede estar asociado a la parametrizacion que usan para generar los rayos \textcolor{red}{(Fierro et al. 2013)}. Al asimilar las observaciones de GLM tanto la distribución de los hidrometeoros como la de carga elecétrica se ven mejoradas respecto de la que no asimiló.

\paragraph{Hu et al. 2021} \label{Hu et al. 2021}
\href{https://agupubs.onlinelibrary.wiley.com/doi/abs/10.1029/2021JD034603}{\textbf{Assessment of Storm-Scale Real Time Assimilation of GOES-16 GLM Lightning-Derived Water Vapor Mass on Short Term Precipitation Forecasts During the 2020 Spring Forecast Experiment}}

Se basan en los resultados obtenidos en \nameref{Fierro et al. 2019} y \nameref{Hu et al. 2020} para hacer una evaluación de la metodología durante un período de 5 semanas. Para asimilar usan 3DVAR y asimilan radar (reflectividad, viento radiay y VAD) y rayos del GLM. Los rayos del GLM los acumulan en 10 minutos y generan pseudoobservaciones de $q_{v}$ de 95\% de humedad 3 km por encima del LCL. Todos los días del período arrancan con 1 hora de 3DVAR haciendo ciclos cada 15 minutos asimilando las observaciones de los anteriores 10 y luego 12 horas de pronóstico. Hacen 4 experimentos, el control, el que asimila radar, el que asimila rayos y el que asimila ambos. Al analizar los diagrams de performance de todo el período ven que los pronósticos del control sobrestiman la frecuencia de los umbrales de precipitación y reflectividad y esto se incrementa en las que asimilan, especialmente las que usan GLM. Lo asocian a que la convección al inicio de las corridas está sobrestimada en intensidad y área. Al analizar los campos espaciales se nota claramente que donde se observaron más rayos hay un bias húmedo en la precipitación. En los casos de mayor exceso de PP hay un bian húmedo en los perfiles en la capa donde asimilan rayos.

\paragraph{Liu et al. 2021a} \label{Liu et al. 2021a}
\href{https://www.mdpi.com/2072-4292/13/18/3584}{\textbf{Impact of Lightning Data Assimilation on Forecasts of a Leeward Slope Precipitation Event in the Western Margin of the Junggar Basin}}

Estudian la asimilación de rayos en pendientes de montaña. Si bien el caso que usan no representa una precipitación extrema, se dio en una zona con poca cobertura de radar y bastante árida por lo que generó un fuerte impacto. Hacen 3 experimentos, uno control que corre durante 9 horas, donde las primeras 5 son de spin-up, un segundo experimento que arranca a las 5 horas de asimilando, hace un segundo ciclo de asimilación a la hora siguiente y luego pronostica por 3 horas y un último expermiento que hace un solo ciclo de asimilación a las 6 horas del control y luego pronostica las siguientes 3. Para asimilar los rayos generan pseudo-observaciones de vapor de agua siguiendo la metodologia de \nameref{Liu et al. 2020} (acumulan rayos durante 1 horas centrados en la hora del análisis, en las zonas donde la humedad del background es menor a 90\% generan obs con ese valor entre el LCL y 3 km encima de este. En los análisis ambos experimentos que asimilan presentan mayor humedad que el control y el que hizo 2 ciclos presenta algo más que el de 1 solo. A la primer hora de pronóstico ambos experimentos que asimilan ven bastante más humedad que el control y presentan regiones de ascensos que no se encuentran en el control. Los hidrometeoros también se encuentran en mayor cantidad en los experimentos que asimilan, especialmente en el que hizo 2 ciclos. En los campos de precipitación a 3 hs los experimentos que asimilan ven claramente un máximo de precipitación donde se dio el evento que coincide con las observaciones pero es un poco más intenso. En regiones cercanas donde hay la precipitación observada es leve el impacto de la asimilación es casi nulo. A nivel general, el experimento que hace 2 ciclos es el que presenta mejores resultados.

\paragraph{Liu et al. 2021b} \label{Liu et al. 2021b}
\href{https://www.mdpi.com/2072-4292/13/16/3090}{\textbf{Assimilating FY-4A Lightning and Radar Data for Improving Short-Term Forecasts of a High-Impact Convective Event with a Dual-Resolution Hybrid 3DEnVAR Method}}

Proponer atacar los problemas de que usar pseudo-observaciones de vapor de agua requiere un tiempo de spin-up para generar la convección, reducir el bias húmedo en los pronósticos y que añadir vapor de agua implica cambios en los campos térmicos y dinámicos usando datos de radar y de rayos del LMI (el GLM chino) usando 3DEnVAR. Para asimilar los rayos generan pseudo-observaciones de 90\% de humedad relativa en los puntos donde hay rayos en el LCL y el tope de la nube estimado por el satélite. Donde la humedad ya es mayor al 90\% no hacen nada. Hacen 2 experimentos, uno asimilando cada 1 hora y otro con ciclos de 15 minutos, en el primero acumulan 1h de rayos centrados en la hora del análisis y en el segundo en los 15 minutos previos. Para el determinístico usan una resolución de 3 km y la microfísica de Thompson y para el ensamble usan los 21 miembros del GEFS y 3 combinaciones de parametrizaciones teniendo así 63 miembros. Llegan a la conclusión de que los mejores resultados se dan usando tanto radar como rayos usando un peso de 0.6 para el ensamble y de 0.4 para el 3DVAR. Al usar esta combinación logran que no se ingrese tanto vapor de agua asociado a los rayos reduciendo el bias positivo de precipitación y eliminan parcialmente la convección espúrea de la corrida control que no asimila.

\paragraph{Torcasio et al. 2021} \label{Torcasio et al. 2021}
\href{https://www.mdpi.com/2072-4292/13/4/682}{\textbf{Impact of Lightning Data Assimilation on the Short-Term Precipitation Forecast over the Central Mediterranean Sea}}

Analizan el impacto de la asimilación de rayos comparando la precipitación pronosticada contra campos de IMERG y observaciones en 6 islas del Mediterraneo. La comparación la hacen durante 30 dias haciendo pronóstico corriendo 2 experimentos, uno que asimila y otro que no. El que asimila lo hace usando la metodología de \nameref{Fierro et al. 2012}. Para todo el período de análisis la precipitación media en ambas configuraciones en inferior a la estimada por IMERG, especialmente sobre agua (Ojo, recordar que Paula H dice que el IMERG sobrestima sobre el agua) pero la que asimila es mayor. Hacen también una comparación con la precipitación observada en 6 islas en donde se nota una mejora en los pronósticos al asimilar.

\paragraph{Xiao et al. 2021a} \label{Xiao et al. 2021a}
\href{https://journals.ametsoc.org/view/journals/mwre/149/2/mwr-d-19-0396.1.xml}{\textbf{Lightning Data Assimilation Scheme in a 4DVAR System and Its Impact on Very Short-Term Convective Forecasting}}
Desarrolan una metodología para asimilar velocidad vertical a partir de observaciones de rayos mediante 4DVAR. La relación que usan para pasar de rayos a velocidad vertical máxima es la de relación de \textcolor{red}{Price y Rind 1992}. Luego expanden la velocidad máxima a un perfil de velocidades a partir de obtener los perfiles de velocidad vertical estimados por el análisis empleado en zonas donde la reflectividad es mayor a 18 dBZ durante una temporada de verano, que normalizan por el máximo y promedian. Luego, como la ascendente no es puntual, aplican también un suavizado horizontal para generar perfiles de viento en el entorno de las observaciones de rayos. Para evaluar la metodología la aplican en un caso de estudio de un MCS que el pronóstico operativo no logró ver bien. Hacen 6 experimentos, 1 control, uno que asimila radar, uno con rayos, otro con ambos y otros 2 con ambos pero variando parámetros. Hacen 9 ciclos de asimilación y desde el 6 al 9 sacan pronósticos a 2 horas. En los análisis de cada experimento se puede ver que cuanta más información asimilan mejor mayores velocidad verticales se obtienen. Por si solo el experimento que solo asimila rayos no logra ubicar bien las convergencias (ni tapoco su estructura vertical, es más bien vertical cuando los que asimilan radar está inclinada) pero si las intensidades el viento, la mejor combinación se da cuando se asimila radar tambien. Notan un bias húmedo en el análisis al asimilar rayos al igual que otros trabajos. Al evaluar los pronósticos concluyen que los rayos ayudan en la intensidad de las tormentas y el radar en el tipo de convección. Los experimentos de sensibilidad los usan para evaluar el vertical de viento vertical de viento empleado, la cantidad de puntos del entorno en donde apĺican la asimilación, la cantidad de tiempo que acumulan rayos y la parametrización para pasar de rayos a velocidad vertical.

\paragraph{Xiao et al. 2021b} \label{Xiao et al. 2021b}
\href{https://www.mdpi.com/2072-4292/13/11/2084}{\textbf{Evaluating the Performance of Lightning Data Assimilationfrom BLNET Observations in a 4DVAR-Based Weather Nowcasting Model for a High-Impact Weather over Beijing}}

Usan la metodología desarrollada en \nameref{Xiao et al. 2021a} que se basa en actualizar la velocidad vertical dependiendo de la actividad eléctrica usando 4DVAR. Interpolan los rayos a la reticula del modelo y acumulan en el tiempo. Para no ajustar mucho en nubes no precipitantes buscan también que la reflectividad de radar sea mayor a 18dBZ en las reticulas adyacentes y +- 6 minutos. Obtienen la tasa de rayos del modelo en función de la velocidad vertical máxima de la columna. Como eso es solo un dato hacen una climatología de la distribución de la velocidad vertical para sacar un perfil. Asimilar observaciones de rayos de la red que montaron en combinación con datos de radar aumenta mucho el skill del pronóstico incluso respecto de una red con menor eficiencia al comaprar contra campos de radar y estimaciones de precipitación. La mejor configuración que encuentran es que la asimilación tenga un radio de influencia de 5 puntos de retícula y acumulando rayos por 3 minutos. No hacen nada sobre la convección espúrea que genera el modelo.

\paragraph{Parodi et al. 2022} \label{Parodi et al. 2022}
\href{https://www.researchgate.net/publication/357768837_A_Nowcasting_Model_for_Severe_Weather_Events_at_Airport_Spatial_Scale_The_Case_Study_of_Milano_Malpensa}{\textbf{A Nowcasting Model for Severe Weather Events at Airport Spatial Scale: The Case Study of Milano Malpensa}}

Evalúan el impacto de la asimilación de datos en un evento que afectó el tráfico aéreo en un aeropuerto de Italia. Asimilan radar (reflectividad) y GNSS (ZTD) en ciclos de 3 horas usando 3DVAR y rayos usando nudging. El paper se centra mucho en el sistema de visualización que desarrollaron. Encuentran que la corrida que asimila radar+rayos es la que mejor ubica la convección pero tiende a sobrestimar el área de precipitación.
