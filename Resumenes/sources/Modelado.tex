% !TEX root = ./main.tex

\chapter{Modelado}

\section{Rayos}

\paragraph{Price y Rind 1992} \label{Price y Rind 1992}
\href{https://agupubs.onlinelibrary.wiley.com/doi/abs/10.1029/92JD00719}{\textbf{A Simple Lightning Parameterization for Calculating Global Lightning Distributions}}

Partiendo de la ecuación del voltaje asociado a una carga puntual llegan a que la energía eléctrica de una tormenta se relaciona con la quinta potencia de su altura. Una relación similar se observa entre la altura de la nube y la tasa de rayos. Para chequear que esta relación se replica para todo el globo toman observaciones de satélite de tope de nube y la pasan a rayos para comparar con observaciones de estos y ver si coinciden. Encuentran que sobrestiman sobre el océano entonces usan la relación entre la altura de al nube y el valor de la ascendente sobre océanos y tierra para generar una parametrización de velocidad vertical a rayos. Al comparar contra las observaciones encuentran que la distribución espacial se asemeja bastante con estadísticos mejores que los del azar; en cuanto a la intensidad, la parametrización explica el 30\% de la varianza de los datos pero creen que podría ser mayor si la resolución de los datos fuera mayor, si realizaran cambios en la forma de verificación y si el sensor que usaron para las observaciones tuviera mayor eficiencia de detección.

\paragraph{Mansell et al. 2002} \label{Mansell et al. 2002}
\href{https://agupubs.onlinelibrary.wiley.com/doi/full/10.1029/2000JD000244}{Simulated three-dimensional branched lightning in a numerical thunderstorm model}

Realizan una parametrización que puede hacer una representación tridimensional de los rayos en un modelo meteorológico. A partir de modelos no meteorológicos de descargas extienden la idea a una tormenta. Tienen que hacer muchas suposiciones, como que el ancho del rayo es igual al tamaño de la retícula y que si un rayo llega se propaga lo suficiéntemente cerca de la superficie (1.75 km AGL) lo consideran que fue a tierra. Realizan una prueba para un caso simulado a partir de un sondeo y hodógrafa típica de una supercelda en el que tienen que parametrizar los procesos de la generación de las cargas eléctricas en la tormenta. Muestran como ejemplo 2 rayos intranube y 2 rayos a tierra, uno positivo y otro negativo. Encuentran que los rayos se inician principalmente cerca de la zona de límite entre las cargas positivas y negativas (máximo potencial). Además destacan varias similitudes entre los resultados obtenidos y los observados en la naturaleza

\paragraph{Mansell et al. 2005} \label{Mansell et al. 2005}
\href{https://agupubs.onlinelibrary.wiley.com/doi/10.1029/2004JD005287}{\textbf{Charge structure and lightning sensitivity in a simulated multicell thunderstorm}}

Van a estudiar la sensibilidad de un modelo que resuelve explícitamente la electrificación de tormentas a la utilización de 5 parametrizaciones distintas de los procesos no inductivos de carga que se basan en estudios que llegaron a conclusiones diferentes (describen cada una), a la utilización o no de un esquema para los procesos inductivos y a la mínima concentración de cristales de hielo. Para parametrizar los rayos usan \nameref{Mansell et al. 2002}. La tormenta simulada la inician a partir de generar perturbaciones de temperatura en un entorno asociado a convección multicelular. En líneas generales encuentran que apagar los procesos inductivos generan más rayos que si estos se encuentran con intensidad moderada pero si se utilizan con mucha intensidad la cantidad de rayos es mucho menor, además de que la cantidad de rayos aumenta al aumentar el mínimo permitido de concentración de cristales de hielo, por este motivo destacan que sería conveniente usar una parametrización de 2 momentos.

\paragraph{Mansell et al. 2010} \label{Mansell et al. 2010}
\href{https://journals.ametsoc.org/view/journals/atsc/67/1/2009jas2965.1.xml}{\textbf{Simulated Electrification of a Small Thunderstorm with Two-Moment Bulk Microphysics}}

Como en \nameref{Mansell et al. 2005} resaltan la importancia de tener usar una microfísica de 2 momentos, en este trabajo usan una de ese tipo. Para las simulaciones usan el modelo Collaborative Model for Multiscale Atmospheric Simulation (COMMAS) con una reoslución de 250 m en la horizontal y 125 m en la vertical. Aunque la parametrización de la microfísica pronostica granizo, no lo consideraron para reducir el costo computacional ya que en una prueba que hicieron no afectaba mucho su exclusión. Describen los cambios realizados a la parametrización de \nameref{Mansell et al. 2002} que emplearon en esta simulación. En este caso el mecanismo inductivo no fue muy relevenate como en \nameref{Mansell et al. 2005} por haber usado una parametrización de 2 momentos.

\paragraph{Yair et al. 2010} \label{Yair et al. 2010}
\href{https://agupubs.onlinelibrary.wiley.com/doi/full/10.1029/2008JD010868}{\textbf{Predicting the potential for lightning activity in Mediterranean storms based on the Weather Research and Forecasting (WRF) model dynamic and microphysical fields}}

Emplean/definen el Lightning Potential Index (LPI) como una variable que incluye variables de la microfísica para pronosticar la presencia de rayos en los modelos. Representa la energía cinética de la ascendente escalada por un potencial de que se produzca la separación de cargas en entre los 0 y -20°C y adquiere un máximo cuando el contenido de agua líquida es igual al de hielo. Evalúan este índice para 5 casos de precipitación extrema en el Mediterraneo comparándolo con índices clásicos como el K index. En todos los casos el LPI logra un mejor desempeño que los índices clásicos para detectar las zonas de tormenta.

\paragraph{Lynn y Yair 2010} \label{Lynn y Yair 2010}
\href{https://adgeo.copernicus.org/articles/23/11/2010/}{\textbf{Prediction of lightning flash density with the WRF model}}

\nameref{Yair et al. 2010} calculó el LPI usando el WRF con una resolución de 1 km, en este trabajo lo evalúan a 4 km pensando en que esta resolución es comunmente empleada en los pronósticos operativos. Simulan 2 casos y en ambos evalúan la relación entre el LPI y la cantidad de rayos observada. En los 2 casos encontraron una alta correlación entre LPI y las observaciones e incluso una relación lineal explica mucho del comportamiento, aunque en uno de los casos una ajuste cuadrático andaba mejor. No encuentran un degradamiento en el LPI al usar la retícula de 4 km.

\paragraph{Fierro y Reisner 2011} \label{Fierro y Reisner 2011}
\href{https://journals.ametsoc.org/view/journals/atsc/68/3/2010jas3659.1.xml}{\textbf{High-Resolution Simulation of the Electrification and Lightning of Hurricane Rita during the Period of Rapid Intensification}}

La tasa de rayos en el ojo de los huracanes es muy baja exceptuando los momentos previos a fases de rápida intensificación. En este trabajo se proponen investigar qué tan bien ven los modelos los eventos convectivos que derivan en la rápida intensificación del huracán Rita. Hacen una simulación con 2 km de resolución para la cual inicializan a partir de una de 4 km en la que asimilan además de otras variables, la actividad eléctrica via nudging a partir de saturar entre 3 y 11 km de altura los puntos de retícula donde se observaron rayos. Logran ver en las simulaciones un incremento de la actividad eléctrica y los eventos convectivos previo a la intensificación del huracán. Hacen correlaciones entre la actividad eléctrica pronosticada por la simulación y distintas variables de pronóstico y teorizan con asimilar la actividad eléctrica usando nudging al saturar capas de atmósfera o añadiendo calor latente.

\paragraph{Lynn et al. 2012} \label{Lynn et al. 2012}
\href{https://journals.ametsoc.org/view/journals/wefo/27/6/waf-d-11-00144_1.xml}{\textbf{Predicting Cloud-to-Ground and Intracloud Lightning in Weather Forecast Models}}

Buscan parametrizar los procesos de carga y descarga de la nubes. Generan una nueva variable en el WRF que es el potencial eléctrico el cual se genera a partir del valor de LPI estimado y cuando supera cierto umbral se descarga. Este potencial eléctrico lo incluyen en la dinámica del modelo entonces puede ir siendo advectado por ejemplo en cada paso del modelo pero los procesos de carga y descarga pueden setearse para correr cada cierta cantidad de pasos. Una simplificación que hacen es que siempre asumen un dipolo típico para las cargas en la nube (cargas negativas en las capas bajas y positivas en las altas). Evalúan la parametrización en varios casos de estudio y probando diferentes combinaciones contra observaciones de 2 fuentes diferentes. En general los pronósticos matchean bien las observaciones en cuanto a la cantidad de rayos y hacen la prueba de eliminar la advección del potencial eléctrico y en ese caso la cantidad de rayos les da menor especialmente para las tasas bajas. Hacen pruebas inicializando los pronoticos con RUC, RAP y GFS, usando parametrizaciones de la microfísica que incluyen granizo o no y con cambios en la resolución de la retícula.

\paragraph{Fierro et al. 2013} \label{Fierro et al. 2013}
\href{https://journals.ametsoc.org/view/journals/mwre/141/7/mwr-d-12-00278.1.xml}{\textbf{The Implementation of an Explicit Charging and Discharge Lightning Scheme within the WRF-ARW Model: Benchmark Simulations of a Continental Squall Line, a Tropical Cyclone, and a Winter Storm}}

Implementan en el WRF un modelo bulk de electrificación, al ser bulk debería ser poco costoso computacionalmente, ellos dicen que es "inexpensive". Implementan el modelo de descargas de \textcolor{red}{MacGorman et al. 2001}. Prueban el modelo en 3 casos, uno de un MCS en el que el modelo logró ver bien la convección, otro de una tormenta invernal en la que la actividad eléctrica fue nula como para probarla en un caso en que no hubo rayos y en un huracán. Comparan los rayos pronosticados contra observaciones y 3 parametrizaciones de rayos basadas en el flujo vertical de graupel a -15°C, el contenido total de hielo en la columna y una combinación de ambas.