% !TEX root = ./main.tex

\chapter{Modelado}

\section{Rayos}

\paragraph{Price y Rind 1992} \label{Price y Rind 1992}
\href{https://agupubs.onlinelibrary.wiley.com/doi/abs/10.1029/92JD00719}{\textbf{A Simple Lightning Parameterization for Calculating Global Lightning Distributions}}

Partiendo de la ecuación del voltaje asociado a una carga puntual llegan a que la energía eléctrica de una tormenta se relaciona con la quinta potencia de su altura. Una relación similar se observa entre la altura de la nube y la tasa de rayos. Para chequear que esta relación se replica para todo el globo toman observaciones de satélite de tope de nube y la pasan a rayos para comparar con observaciones de estos y ver si coinciden. Encuentran que sobrestiman sobre el océano entonces usan la relación entre la altura de al nube y el valor de la ascendente sobre océanos y tierra para generar una parametrización de velocidad vertical a rayos. Al comparar contra las observaciones encuentran que la distribución espacial se asemeja bastante con estadísticos mejores que los del azar; en cuanto a la intensidad, la parametrización explica el 30\% de la varianza de los datos pero creen que podría ser mayor si la resolución de los datos fuera mayor, si realizaran cambios en la forma de verificación y si el sensor que usaron para las observaciones tuviera mayor eficiencia de detección.

\paragraph{Mansell et al. 2002} \label{Mansell et al. 2002}
\href{https://agupubs.onlinelibrary.wiley.com/doi/full/10.1029/2000JD000244}{Simulated three-dimensional branched lightning in a numerical thunderstorm model}

Realizan una parametrización que puede hacer una representación tridimensional de los rayos en un modelo meteorológico. A partir de modelos no meteorológicos de descargas extienden la idea a una tormenta. Tienen que hacer muchas suposiciones, como que el ancho del rayo es igual al tamaño de la retícula y que si un rayo llega se propaga lo suficiéntemente cerca de la superficie (1.75 km AGL) lo consideran que fue a tierra. Realizan una prueba para un caso simulado a partir de un sondeo y hodógrafa típica de una supercelda en el que tienen que parametrizar los procesos de la generación de las cargas eléctricas en la tormenta. Muestran como ejemplo 2 rayos intranube y 2 rayos a tierra, uno positivo y otro negativo. Encuentran que los rayos se inician principalmente cerca de la zona de límite entre las cargas positivas y negativas (máximo potencial). Además destacan varias similitudes entre los resultados obtenidos y los observados en la naturaleza

\paragraph{Mansell et al. 2005} \label{Mansell et al. 2005}
\href{https://agupubs.onlinelibrary.wiley.com/doi/10.1029/2004JD005287}{\textbf{Charge structure and lightning sensitivity in a simulated multicell thunderstorm}}

Van a estudiar la sensibilidad de un modelo que resuelve explícitamente la electrificación de tormentas a la utilización de 5 parametrizaciones distintas de los procesos no inductivos de carga que se basan en estudios que llegaron a conclusiones diferentes (describen cada una), a la utilización o no de un esquema para los procesos inductivos y a la mínima concentración de cristales de hielo. Para parametrizar los rayos usan \nameref{Mansell et al. 2002}. La tormenta simulada la inician a partir de generar perturbaciones de temperatura en un entorno asociado a convección multicelular. En líneas generales encuentran que apagar los procesos inductivos generan más rayos que si estos se encuentran con intensidad moderada pero si se utilizan con mucha intensidad la cantidad de rayos es mucho menor, además de que la cantidad de rayos aumenta al aumentar el mínimo permitido de concentración de cristales de hielo.

\paragraph{Fierro y Reisner 2011} \label{Fierro y Reisner 2011}
\href{https://journals.ametsoc.org/view/journals/atsc/68/3/2010jas3659.1.xml}{\textbf{High-Resolution Simulation of the Electrification and Lightning of Hurricane Rita during the Period of Rapid Intensification}}

La tasa de rayos en el ojo de los huracanes es muy baja exceptuando los momentos previos a fases de rápida intensificación. En este trabajo se proponen investigar qué tan bien ven los modelos los eventos convectivos que derivan en la rápida intensificación del huracán Rita. Hacen una simulación con 2 km de resolución para la cual inicializan a partir de una de 4 km en la que asimilan además de otras variables, la actividad eléctrica via nudging a partir de saturar entre 3 y 11 km de altura los puntos de retícula donde se observaron rayos. Logran ver en las simulaciones un incremento de la actividad eléctrica y los eventos convectivos previo a la intensificación del huracán. Hacen correlaciones entre la actividad eléctrica pronosticada por la simulación y distintas variables de pronóstico y teorizan con asimilar la actividad eléctrica usando nudging al saturar capas de atmósfera o añadiendo calor latente.