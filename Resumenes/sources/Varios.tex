% !TEX root = ./main.tex
% !TEX outputDirectory = ./Build/
\chapter{Varios}

\paragraph{Cheze y Sauvageot 1997} \label{heze y Sauvageot 1997}
\textbf{Area-average rainfall and lightning activity}

Tratan de encontrar una relación entre la precipitación en un área y la actividad eléctrica registrada. Para estimar la precipitación utilizan un PPI bajo de radar y Marshall y Palmer y 2 redes de superficie para estimar la ubicación de la actividad eléctrica (95\% de eficiencia de detección) donde una solo detecta rayos a tierra y la otra trabaj en VHF detectando también los intra-nube. Encuentran que existe una relación entre el área de precipitación que supera cierto umbral y la actividad eléctrica y que es del tipo exponencial. Aunque también encuentran que para algún valor de precipitación del orden de los 10 a 15 $\frac{mm}{hr}$ la relación es lineal porque el exponente de la relación pasa de ser mayor a 1 para valores a bajos a de precipitación a menor a 1 para los valores más altos. La baja relación que encuentran para los umbrales bajos es que entran datos con precipitación estratiforme. Además calculan relaciones entre las descargas a tierra y las totales. Las relaciones que encuentran aplican solo a los casos de estudio que analizan, al juntar todo la relación es mínima o desaparece.

\paragraph{Dietrich et al. 2011} \label{Dietrich et al. 2011}
\textbf{Lightning-based propagation of convective rain fields}

Las mediciones de satélites de microondas permiten tener buenas estimaciones de precipitación pero con la desventaja de que estos tienen pocas pasadas por el mismo sitio entonces se requiere de alguna metodología para rellenar los huecos. Existen metodologías desarrolladas en esta línea pero no funcionan en tiempo operativo porque requieren de tener 2 imágenes, por este motivo desarrollan una metodología para extrapolar los campos de precipitación a partir de rayos. La metodología se basa en generar matrices de traslación a partir de la correlación entre la actividad eléctrica en 2 tiempos consecutivos y otra matriz de modulación que al aplicarlas al campo de precipitación permiten modular y mover los patrones de precipitación observados. Al obtener una tasa de precipitación por rayo detectado se puede también "generar" precipitación. Lo prueban para 2 casos de estudio y en ambos da bien al compararlo con la estimación de precipitación en la siguiente pasada del satélite, incluso en uno que propagan por 5 horas.
