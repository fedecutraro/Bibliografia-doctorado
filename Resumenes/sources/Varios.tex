% !TEX root = ./main.tex
% !TEX outputDirectory = ./Build/
\chapter{Varios}

\paragraph{Dietrich et al. 2011} \label{Dietrich et al. 2011}
\textbf{Lightning-based propagation of convective rain fields}

Las mediciones de satélites de microondas permiten tener buenas estimaciones de precipitación pero con la desventaja de que estos tienen pocas pasadas por el mismo sitio entonces se requiere de alguna metodología para rellenar los huecos. Existen metodologías desarrolladas en esta línea pero no funcionan en tiempo operativo porque requieren de tener 2 imágenes, por este motivo desarrollan una metodología para extrapolar los campos de precipitación a partir de rayos. La metodología se basa en generar matrices de traslación a partir de la correlación entre la actividad eléctrica en 2 tiempos consecutivos y otra matriz de modulación que al aplicarlas al campo de precipitación permiten modular y mover los patrones de precipitación observados. Al obtener una tasa de precipitación por rayo detectado se puede también "generar" precipitación. Lo prueban para 2 casos de estudio y en ambos da bien al compararlo con la estimación de precipitación en la siguiente pasada del satélite, incluso en uno que propagan por 5 horas.
